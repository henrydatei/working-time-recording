% Dieses Dokument muss mit PDFLatex gesetzt werden
% Vorteil: Grafiken koennen als jpg, png, ... verwendet werden
%          und die Links im Dokument sind auch gleich richtig
%
%Ermöglicht \\ bei der Titelseite (z.B. bei supervisor)
%Siehe https://github.com/latextemplates/uni-stuttgart-cs-cover/issues/4
\RequirePackage{kvoptions-patch}

%English:
%\let\ifdeutsch\iffalse
%\let\ifenglisch\iftrue

%German:
\let\ifdeutsch\iftrue
\let\ifenglisch\iffalse

%
\ifenglisch
	\PassOptionsToClass{numbers=noenddot}{scrbook}
\else
	%()Aus scrguide.pdf - der Dokumentation von KOMA-Script)
	%Nach DUDEN steht in Gliederungen, in denen ausschließlich arabische Ziffern für die Nummerierung
	%verwendet werden, am Ende der Gliederungsnummern kein abschließender Punkt
	%(siehe [DUD96, R3]). Wird hingegen innerhalb der Gliederung auch mit römischen Zahlen
	%oder Groß- oder Kleinbuchstaben gearbeitet, so steht am Ende aller Gliederungsnummern ein
	%abschließender Punkt (siehe [DUD96, R4])
	\PassOptionsToClass{numbers=autoendperiod}{scrbook}
\fi

%Warns about outdated packages and missing caption delcarations
%See https://www.ctan.org/pkg/nag
\RequirePackage[l2tabu, orthodox]{nag}

%Neue deutsche Trennmuster
%Siehe http://www.ctan.org/pkg/dehyph-exptl und http://projekte.dante.de/Trennmuster/WebHome
%Nur für pdflatex, nicht für lualatex
\RequirePackage{ifluatex}
\ifluatex
%do not load anything
\else
	\ifdeutsch
		\RequirePackage[ngerman=ngerman-x-latest]{hyphsubst}
	\fi
\fi

\documentclass[
               fontsize=12pt, %Default: 11pt, bei Linux Libertine zu klein zum Lesen
% BEGINN: Optionen für typearea
               paper=a4,
               oneside,  % fuer die Betrachtung am Schirm ungeschickt
               BCOR=3mm, % Bindekorrektur
               DIV=13,   % je höher der DIV-Wert, desto mehr geht auf eine Seite. Gute werde sind zwischen DIV=12 und DIV=15
               headinclude=true,
               footinclude=false,
% ENDE: Optionen für typearea
%               titlepage,
               bibliography=totoc,
%               idxtotoc,   %Index ins Inhaltsverzeichnis
%                liststotoc, %List of X ins Inhaltsverzeichnis, mit liststotocnumbered werden die Abbildungsverzeichnisse nummeriert
               headsepline,
               %cleardoublepage=empty,
               parskip=half,
               draft    % um zu sehen, wo noch nachgebessert werden muss - wichtig, da Bindungskorrektur mit drin
%               final,   % ACHTUNG! - in pagestyle.tex noch Seitenstil anpassen
%Kein Punkt nach letzter Gliederungsebene               
			numbers=noenddot
               ]{scrbook}


\input{preambel/packages_and_options}

%Der untere Rand darf "flattern"
\raggedbottom

%%%
% Wie tief wird das Inhaltsverzeichnis aufgeschlüsselt
% 0 --\chapter
% 1 --\section % fuer kuerzeres Inhaltsverzeichnis verwenden - oder minitoc benutzen
% 2 --\subsection
% 3 --\subsubsection
% 4 --\paragraph
\setcounter{tocdepth}{1}
%
%%%

\makeindex

%Angaben in die PDF-Infos uebernehmen
\makeatletter
\hypersetup{
            pdftitle={Entwicklung eines Business-Intelligence-Systems für die Arbeitszeiterfassung im Hochschulbereich}, %Titel der Arbeit
            pdfauthor={Henry Haustein}, %Author
            pdfkeywords={}, % CR-Klassifikation und ggf. weitere Stichworte
            pdfsubject={}
}
\makeatother

% Hier stehen alle Abkürzungen
\newacronym[plural={Informationssystemen}]{is}{IS}{Informationssystem}
\newacronym{bir}{BIR}{Business Intelligence Research}
\newacronym{slub}{SLUB}{Sächsische Landesbibliothek - Staats- und Universitätsbibliothek Dresden}
\newacronym{tud}{TUD}{Technische Universität Dresden}
\newacronym{dl}{DL}{Deep Learning}


\begin{document}

%tex4ht-Konvertierung verschönern
%\iftex4ht
% tell tex4ht to create picures also for formulas starting with '$'
% WARNING: a tex4ht run now takes forever!
%\Configure{$}{\PicMath}{\EndPicMath}{} 
%$ % <- syntax highlighting fix for emacs
%\Css{body {text-align:justify;}}

%conversion of .pdf to .png
%\Configure{graphics*}  
%         {pdf}  
%         {\Needs{"convert \csname Gin@base\endcsname.pdf  
%                               \csname Gin@base\endcsname.png"}%  
%          \Picture[pict]{\csname Gin@base\endcsname.png}%  
%         }  
%\fi

%Tipp von http://goemonx.blogspot.de/2012/01/pdflatex-ligaturen-und-copynpaste.html
%siehe auch http://tex.stackexchange.com/questions/4397/make-ligatures-in-linux-libertine-copyable-and-searchable
%
%ONLY WORKS ON MiKTeX
%On other systems, download glyphtounicode.tex from http://pdftex.sarovar.org/misc/
%
%\input glyphtounicode.tex
%\pdfgentounicode=1

%\VerbatimFootnotes %verbatim text in Fußnoten erlauben. Geht normalerweise nicht.

\input{macros/commands}
%\pagenumbering{arabic}

% ================= Introduction ===========================

\Titelblatt
%\litzf
\pagenumbering{gobble}
%Eigener Seitenstil fuer die Kurzfassung und das Inhaltsverzeichnis
\deftripstyle{preamble}{}{}{}{}{}{\pagemark}
%Doku zu deftripstyle: scrguide.pdf
\pagestyle{preamble}
\renewcommand*{\chapterpagestyle}{preamble}

%Kurzfassung / abstract
%auch im Stil vom Inhaltsverzeichnis
\section*{Abstract}
This paper shall act as a collection of guidlines in order to support scientific writing for the purpose of creating term papers, a master as well as a bachelor or diploma thesis. Nevertheless this paper does not act as a substitue for dealing with literature about scientific writing and the process of claiming knowledge in that particular area. Apart from freedom of science there are certain formal requirements stated within this guide, that need to be fullfilled in order for the paper to be accepted at the chair of Business Intelligence Research.

\cleardoublepage


% BEGIN: Verzeichnisse

%\iftex4ht
%\else
%\microtypesetup{protrusion=false}
%\fi

%%%
% Literaturverzeichnis ins TOC mit aufnehmen, aber nur wenn nichts anderes mehr hilft!
% \addcontentsline{toc}{chapter}{Literaturverzeichnis}
%
% oder zB
%\addcontentsline{toc}{section}{Abkürzungsverzeichnis}
%
%%%

%Produce table of contents
%
%In case you have trouble with headings reaching into the page numbers, enable the following three lines.
%Hint by http://golatex.de/inhaltsverzeichnis-schreibt-ueber-rand-t3106.html
%
%\makeatletter
%\renewcommand{\@pnumwidth}{2em}
%\makeatother
%

%Seitennummerierung in römischen Zahlen Für alles vor dem Text
\setcounter{page}{1} 
\pagenumbering{Roman}

\tableofcontents

% Bei einem ungünstigen Seitenumbruch im Inhaltsverzeichnis, kann dieser mit
% \addtocontents{toc}{\protect\newpage}
% an der passenden Stelle im Fließtext erzwungen werden.

\listoffigures
\ifdeutsch
\addcontentsline{toc}{chapter}{\normalsize{Abbildungsverzeichnis}}
\else
\addcontentsline{toc}{chapter}{\normalsize{List of Figures}}
\fi

\listoftables
\ifdeutsch
\addcontentsline{toc}{chapter}{\normalsize{Tabellenverzeichnis}}
\else
\addcontentsline{toc}{chapter}{\normalsize{List of Tables}}
\fi

%Wird nur bei Verwendung von der lstlisting-Umgebung mit dem "caption"-Parameter benoetigt
%\lstlistoflistings 
%ansonsten:
\ifdeutsch
\listof{Listing}{Verzeichnis der Listings}
\addcontentsline{toc}{chapter}{\normalsize{Verzeichnis der Listings}}
\else
\listof{Listing}{List of Listings}
\addcontentsline{toc}{chapter}{\normalsize{List of Listings}}
\fi

%mittels \newfloat wurde die Algorithmus-Gleitumgebung definiert.
%Mit folgendem Befehl werden alle floats dieses Typs ausgegeben
\ifdeutsch
\listof{Algorithmus}{Verzeichnis der Algorithmen}
\addcontentsline{toc}{chapter}{\normalsize{Verzeichnis der Algorithmen}}
\else
\listof{Algorithmus}{List of Algorithms}
\addcontentsline{toc}{chapter}{\normalsize{List of Algorithms}}
\fi
%\listofalgorithms %Ist nur für Algorithmen, die mittels \begin{algorithm} umschlossen werden, nötig

% Abkürzungsverzeichnis
\printnoidxglossaries
\ifdeutsch
\addcontentsline{toc}{chapter}{\normalsize{Abkürzungsverzeichnis}}
\else
\addcontentsline{toc}{chapter}{\normalsize{List of Abbreviations}}
\fi

\iftex4ht
\else
%Optischen Randausgleich und Grauwertkorrektur wieder aktivieren
\microtypesetup{protrusion=true}
\fi

% END: Verzeichnisse


\renewcommand*{\chapterpagestyle}{scrplain}
\pagestyle{scrheadings}
\input{preambel/pagestyle}

% ====== Main Part ===============
%KS: Seitennummerierung des Hauptteils
\mainmatter
%KS: Nummerierung beginnt beim Hauptteil neu mit arabischen Ziffern
\pagenumbering{arabic}
\setcounter{page}{1}
%
%
% ** Hier wird der Text eingebunden **
%
\chapter{Allgemeine Angaben}
\label{chap:allgemeine_Angaben}

Die folgenden Angaben beziehen sich auf Vorgaben des Lehrstuhls für
Wirtschaftsinformatik | Business Intelligence Research der Technischen Universität Dresden.\\
Das folgende Kapitel beschäftigt sich dabei im Speziellen mit der Gliederung einer wissenschaftlichen Arbeit und den einzelnen Verzeichnissen.

\section{Gliederung}\index{Gliederung}
\label{chap:gliederung_Arbeit}

Eine wissenschaftliche Abschluss- oder Seminararbeit sollte wie folgt gegliedert sein:

\begin{itemize}
\item Einleitung
\item Kapitel 1
\item . . .
\item Kapitel n
\item Fazit (Zusammenfassung und Ausblick)
\item Literaturverzeichnis
\item Anhang
\end{itemize}

Die \textbf{Einleitung}\index{Einleitung} soll das zu behandelnde Thema vorstellen und in den übergeordneten
Sachzusammenhang einordnen. Sie soll den Leser zum Lesen der Arbeit motivieren und Ziele, Inhalte und Ergebnisse der
einzelnen Kapitel skizzenhaft vorstellen. 

Der \textbf{Hauptteil}\index{Hauptteil} der Arbeit ist in mehrere Kapitel zu
unterteilen, die dann ihrerseits wiederum verschiedene Abschnitte und Unterabschnitte (evtl. mit
Mehrfachuntergliederungen) aufweisen können. Die Kapiteleinteilung muss systemlogisch angelegt sein,
d.~h., kein Punkt oder Unterpunkt kann alleine (ohne einen weiteren korrespondierenden,
gleichgeordneten Punkt) stehen. Siehe dazu die Nummerische Gliederung nach dem Abstufungsprinzip (vgl.
\cite{Theisen2006}, S.~102~f.). Außerdem sollten die einzelnen Punkte (Kapitel) vom Umfang her gleichumfänglich sein,
um eine \glqq Klumpenbildung\grqq \ zu vermeiden. 

Im \textbf{Fazit}\index{Fazit} werden die Ergebnisse der Arbeit noch
einmal zusammengefasst, ggf. weiterführende oder angrenzende Problemstellungen aufgezeigt bzw. diskutiert. Weiterhin
kann auf nicht behandelte aber als relevant erachtete Problemstellungen hingewiesen werden.

\section{Verzeichnisse}\index{Verzeichnis}
\label{chap:verzeichnisse}
Werden in der Arbeit viele verschiedene Abkürzungen verwendet, so sollten diese -- auch im Falle von Zeitschriften --
in einem \textbf{Abkürzungsverzeichnis}\index{Verzeichnis!Abkürzungs-}\index{Abkürzungsverzeichnis} aufgelistet und
kurz erklärt werden. Hierbei werden Abkürzungen, welche im Duden zu finden sind (\glqq z.~B.\grqq , \glqq s.\grqq, etc.) nicht aufgelistet.
 Ebenso müssen ein
\textbf{Abbildungs}\index{Verzeichnis!Abbildungs-}\index{Abbildungsverzeichnis}-
und \textbf{Tabellenverzeichnis}\index{Tabellenverzeichnis}\index{Tabellenverzeichnis} (unter Angabe der jeweils
zugehörigen Seitenzahlen) vorhanden sein, wenn in die Arbeit Abbildungen und/oder Tabellen eingearbeitet sind.
In den meisten Fällen schließt die Arbeit mit dem
\textbf{Literaturverzeichnis}\index{Verzeichnis!Literatur-}\index{Literaturverzeichnis} ab. Bei Verwendung von vielen
aufeinander folgenden oder umfangreichen Tabellen und Abbildungen oder einer ausführlichen formelmäßigen Herleitung
(z.~B. zur Klärung oder Verdeutlichung einer bestimmten Aussage) sowie ergänzenden Materialien und Dokumenten in der
Arbeit sollten die betreffenden Teile in einem \textbf{Anhang}\index{Anhang} (im Anschluss an das
Literaturverzeichnis) aufgeführt werden, so dass dadurch der Lesefluss nicht gestört wird. Der Anhang kann (wenn dies
inhaltlich sinnvoll erscheint) in mehrere Teile gegliedert sein. Die einzelnen Kapitel und Abschnitte der Arbeit sind
unter Angabe der Seitenzahl in einem \textbf{Inhaltsverzeichnis}\index{Inhaltsverzeichnis}\index{Verzeichnis!Inhalts-}
aufzulisten, das der eigentlichen Arbeit vorangestellt wird.\\

Die gesamte Arbeit (einschließlich Einleitung und Literaturverzeichnis) muss mit einer
\textbf{Seitennummerierung}\index{Seitennummerierung}\index{Nummerierung!Seiten-} bzw. Paginierung\index{Paginierung}
versehen werden. Dabei ist nur der Textteil mit fortlaufenden \textbf{arabischen Ziffern}\index{Ziffern!arabisch} zu
nummerieren, und die restlichen Teile der Arbeit (Verzeichnisse und Anhang) sind wiederum fortlaufend mit
\textbf{römischen Ziffern}\index{Ziffern!römisch} zu nummerieren, wobei das Literaturverzeichnis dem Textteil zuzurechnen ist (vgl.
\cite{Theisen2006}, S.~179~f.).\\

\section{Exposé}
\label{chap:expose}
Vor Beginn des Schreibens der Arbeit ist mit dem jeweiligen Betreuer das Vorgehen für die Bearbeitung der Arbeit abzuklären. Dazu muss nach Ausgabe des Themas in der Regel binnen zwei Wochen ein Exposé erstellt werden, in dem dargestellt wird, welche Ziele mit welchen Forschungsmethoden verfolgt werden (Forschungsdesign). Zu Informationen zum Forschungsdesign und der allgemeinen Wissenschaftstheorie siehe Anhang \ref{chap:Wi_Th}.
Die Bestandteile des Exposés sind ausführlich und mit einem Beispiel im Anhang \ref{chap:a_expose} beschrieben. Das Exposé ist mit dem jeweiligen Betreuer zu besprechen und dient in seiner abgesegneten Form als Forschungsgrundlage für die anzufertigende Arbeit.



\section{Eidestattliche Erklärung}
\label{chap:diplomarbeiten}

Bei Diplom-, Studien- und Seminararbeiten muss eine \textbf{Eidesstattliche
Erklärung}\index{Erklärung!Eidesstattliche}\index{Eidesstattliche Erklärung} am Ende der Arbeit eingefügt werden. Der
genaue Wortlaut für Arbeiten mit WORD ist der Vorlage im Anhang \ref{eides} zu entnehmen. Bei der
\LaTeX-Vorlage wird die Eidesstattliche Erklärung bei Benutzung des entsprechenden Argumentes automatisch erzeugt. 
%\input{...weitere Kapitel...}

%KS: Literaturverzeichnis nach dem Haupt-Textteil
\printbibliography
\clearpage
%
%
% ======== Appendix ==================

%Seitennummerierung in römischen Zahlen - Anpassen auf die jeweilige Seitenzahl+1 vor dem Haupt-Textteil
\pagenumbering{Roman}
\setcounter{page}{11}
\appendix
\renewcommand{\appendixtocname}{Anhang}
\renewcommand{\appendixname}{Anhang}
\renewcommand{\appendixpagename}{Anhang}
\appendixpage
%%\begin{appendices}
\begin{appendix}
%\renewcommand{\thesection}{\Alph{section}}
\chapter{Ehrenwörtliche Erklärung}\label{eides}
\thispagestyle {empty}

\begin{tabular}{ll}
  Name: & Mustermann \\
  Vorname: & Max \\
  Matrikel-Nr.: & 1234567 \\
\end{tabular}

\vspace{15mm}

\begin{center}\underline{Eidesstattliche Erklärung}\end{center}
Ich erkläre hiermit, dass ich die vorliegende Arbeit
selbstständig verfasst und keine anderen als die angegebenen Quellen
und Hilfsmittel benutzt habe, dass alle Stellen der Arbeit, die
wörtlich oder sinngemäß aus anderen Quellen übernommen wurden, als
solche kenntlich gemacht sind, und dass die Arbeit in gleicher
oder ähnlicher Form noch keiner Prüfungsbehörde vorgelegt wurde.\\

\vspace{30mm}

Dresden, aktuelles Datum \hspace{35mm} \hrulefill \\
\hspace*{83mm} Unterschrift
\end{appendix}
%%\end{appendices}
\clearpage

%\printindex



%\ifdeutsch
%Alle URLs wurden zuletzt am 17.\,03.\,2008 geprüft.
%\else
%All links were last followed on March 17, 2008.
%\fi

\pagestyle{empty}
\renewcommand*{\chapterpagestyle}{empty}
%\Versicherung
\end{document}
